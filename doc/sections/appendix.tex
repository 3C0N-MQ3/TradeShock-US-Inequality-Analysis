\begin{frame}
    \centering
    \Huge\bfseries Appendix
\end{frame}

% ------------------------------------------------------------------------

\begin{frame}
    \begin{table}[ht]
        \centering
        \caption{Imports from China and Employment Status of Working-Age Population within CZs, 1990--2007: 2SLS Estimates}
        \label{tab:table_5}
        \resizebox{1\textwidth}{!}{
            \begin{tabular}{lcccccccccc}
                \toprule
                \multicolumn{9}{c}{Dependent variables: Ten-year equivalent changes in log population counts and population shares by employment status}&&\\
                \cmidrule(lr){1-11}
                & \multicolumn{2}{c}{Mfg emp} & \multicolumn{2}{c}{Non-mfg emp} & \multicolumn{2}{c}{Unemp} & \multicolumn{2}{c}{NILF} \\
                & \multicolumn{2}{c}{(1)}     & \multicolumn{2}{c}{(2)}         & \multicolumn{2}{c}{(3)}   & \multicolumn{2}{c}{(4)}  \\
                \cmidrule(lr){2-3} \cmidrule(lr){4-5} \cmidrule(lr){6-7} \cmidrule(lr){8-9}
                & Original & CA  & Original & CA  & Original & CA  & Original & CA \\
                \midrule
                \multicolumn{9}{l}{\textit{Panel A. 100 $\times$ log change in population counts}} \\
$(\Delta \text{ imports from China to US})/\text{worker}$ 
                & -4.231*** & -4.831*** & -0.274     & -0.013     & 4.921*** & 5.884*** & 2.058*   & 3.296*   \\
                & (1.047)   & (1.215)   & (0.651)    & (0.667)    & (1.128)  & (1.138)  & (1.080)  & (1.103)  \\
                \midrule
                \multicolumn{9}{l}{\textit{Panel B. Change in population shares}} \\
\textit{All education levels} \\
$(\Delta \text{ imports from China to US})/\text{worker}$ 
                & -0.596*** & -0.640*** & -0.178     & -0.118     & 0.221*** & 0.218*** & 0.553*** & 0.539*** \\
                & (0.099)   & (0.120)   & (0.137)    & (0.117)    & (0.058)  & (0.053)  & (0.150)  & (0.121)  \\
\textit{College education} \\
$(\Delta \text{ imports from China to US})/\text{worker}$ 
                & -0.592*** & -0.510*** & 0.168      & 0.165      & 0.119*** & 0.052**  & 0.304*** & 0.293*** \\
                & (0.125)   & (0.159)   & (0.122)    & (0.156)    & (0.039)  & (0.025)  & (0.113)  & (0.084)  \\
\textit{No college education} \\
$(\Delta \text{ imports from China to US})/\text{worker}$ 
                & -0.581*** & -0.640*** & -0.531***  & -0.265*    & 0.282*** & 0.268*** & 0.831*** & 0.638*** \\
                & (0.095)   & (0.107)   & (0.203)    & (0.146)    & (0.085)  & (0.070)  & (0.211)  & (0.160)  \\
                \bottomrule
            \end{tabular}
        }
        \vspace{0.2cm}
        
        \begin{minipage}{\linewidth}
            \tiny
            \textit{Notes:} $N = 1,444$ (722 CZs $\times$ two time periods). All statistics are based on working-age individuals (ages 16 to 64). The effect of import exposure on the overall employment/population ratio can be computed as the sum of the coefficients for manufacturing and non-manufacturing employment. This effect is highly statistically significant ($p \leq 0.01$) in the full sample and in all reported subsamples, for both composition-adjusted and non-adjusted models. All regressions include the full vector of control variables from column 6 of Table \ref{tab:table_3}. Robust standard errors in parentheses are clustered on state. Models are weighted by start of period CZ share of national population. \\
            ***Significant at the 1 percent level. \\
            **Significant at the 5 percent level. \\
            *Significant at the 10 percent level.
        \end{minipage}
    \end{table}
\end{frame}