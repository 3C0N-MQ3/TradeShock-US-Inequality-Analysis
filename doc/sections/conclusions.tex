\begin{frame}
    \frametitle{Conclusions}

    \begin{itemize}
        \item Our analysis, using composition-adjusted Two-Stage Least Squares (2SLS) models, highlights a more substantial impact of the China trade shock on regional inequality across U.S. labor markets.
        
        \begin{itemize}
            \item \textbf{Stronger Decline in Manufacturing Employment}: Composition-adjusted results indicate a sharper drop in manufacturing employment in response to increased import exposure, reinforcing the vulnerability of manufacturing-dependent regions to trade shocks.
            
            \item \textbf{More Pronounced Wage Declines}: The adjusted analysis reveals that wage reductions, especially among workers without a college education, are more significant than previously estimated. This impact is heightened by accounting for demographic shifts, underscoring the differential effect on lower-educated workers.
            
            \item \textbf{Higher Unemployment and Non-Participation Rates}: The adjusted model shows a significant increase in unemployment and non-labor force participation within regions exposed to import competition. This trend is less apparent in the unadjusted results, demonstrating the importance of composition adjustments to accurately capture labor market displacement.
        \end{itemize}
        
        \item In summary, our findings contrast with the unadjusted model by illustrating that composition adjustments amplify the observed effects, leading to a deeper understanding of the China trade shock’s role in driving regional inequality.
    \end{itemize}

\end{frame}
