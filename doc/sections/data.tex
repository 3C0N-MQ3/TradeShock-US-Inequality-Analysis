\begin{frame}
    \frametitle{Data Description}
    \framesubtitle{Sources}
    The two main data sources used in this study are:
    \begin{enumerate}
        \item Dependent variables from the 1990, 2000 Censuses + 2007 3 year ACS (2006-2008)
        \begin{itemize}
            \item \url{https://usa.ipums.org/usa/}
        \end{itemize}
        \item \textbf{Data II:} Independent variables from “China Syndrome” paper
        \begin{itemize}
            \item \textbf{Author:} David H. Autor, David Dorn, and Gordon H. Hanson, 2013, “The China Syndrome: Local Labor Market Effects of Import Competition in the United States”
            \item \textbf{Dorn Data:} \url{http://www.ddorn.net/data.html}
        \end{itemize}
    \end{enumerate}
\end{frame}
%% ------------------------------------------------------------------------ 
\begin{frame}
    \frametitle{Data Description}
    \framesubtitle{Transformations}
    \begin{itemize}
        \item \textit{Step 1:} Ten-year equivalent changes (btw 1990-2000 and btw 2000-2007) are constructed at the commuting zone (CZ) level, composition-adjusted for working-age individuals (also the same for 1990 - 2000) for
        \begin{itemize}
            \item average wage, e.g., log(average wage\textsubscript{2007}) log(average wage\textsubscript{2000})
            \item unemployment rate, e.g., unemployment rate\textsubscript{2007}/unemployment rate\textsubscript{2000}
            \item labor force participation (LFP) rate, e.g., LFP rate\textsubscript{2007}/LFP rate\textsubscript{2000}
        \end{itemize}
        \item \textit{Step 2:} 2SLS is used to estimate the impact of the “China shock” on the CZ-level outcomes above, with progressive controls added for additional CZ-level variables as described in the paper.
    \end{itemize}
\end{frame}



