\begin{frame}

    \begin{block}{Research Question}
        \begin{itemize}
        \item What is the impact of a large international trade shock on inequality across regions in the US?
        \item How the increasing of China’s imports affected the US manufacturing employment?
        \item How those Commuting zones (CZs) that faced strong exposed import competition with China responded to rising Chinese import?
        \end{itemize}
    \end{block}
    
    \begin{block}{Methodological Approach} 
        A composition-adjusted Two-Stage Least Squares regression is employed, with demographic factors (gender, age, education, race, nativity) controlled to isolate shifts that might affect inequality across industries.
    \end{block}
    
    \begin{block}{Key Findings} 
        It is observed that the China shock has amplified regional inequality. By adjusting for composition, a stronger impact of increased Chinese imports on economic variables is revealed, especially on unemployment rates. Workers in industries heavily impacted by import competition are found to face unemployment rather than transitioning to higher-wage sectors.
    \end{block}
    
\end{frame}