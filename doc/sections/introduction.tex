\begin{frame}{Introduction}

    \begin{block}
    
        \textbf{Research Question:} The impact of a significant international trade shock on regional inequality in the US is examined.
    
    \end{block}
    \begin{block}
    
        \textbf{Methodological Approach:} A composition-adjusted Two-Stage Least Squares regression is employed, with demographic factors (gender, age, education, race, nativity) controlled to isolate shifts that might affect inequality across industries.
    
    \end{block}
    \begin{block}
    
        \textbf{Key Findings:} It is observed that the China shock has amplified regional inequality. By adjusting for composition, a stronger impact of increased Chinese imports on economic variables is revealed, especially on unemployment rates. Workers in industries heavily impacted by import competition are found to face unemployment rather than transitioning to higher-wage sectors.
    \end{block}
    
\end{frame}