\begin{frame}{Introduction}

    \begin{block}{Research Question}
        What is the impact of a large international trade shock on inequality across regions in the US?
        How the increasing of China’s imports affected the US manufacturing employment?
        How those Commuting zones (CZs) that faced strong exposed import competition with China responded to rising Chinese import?
    \end{block}

    \begin{block}{Methodologic Approach}
        A composition-adjusting approach is applied to the outcomes using a Two-Stage Least Squares Regression Analysis. Potential changes in the composition of gender, age, education, race, and nativity (US-born) are held constant to assess demographic shifts across industries that may impact inequality.
    \end{block}

    \begin{block}{Key Findings}
        Larger effects of the China shock on inequality across regions are observed.
        Applying the composition-adjusting technique reveals changes in the magnitude of the impact of rising Chinese imports on various U.S. economic variables, with the most notable effect observed in unemployment rates. Workers initially employed in sectors highly exposed to import competition tend not to transition to higher-wage industries; instead, they face Unemployment.
    \end{block}
\end{frame}