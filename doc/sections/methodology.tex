\begin{frame}
    \frametitle{Methodology}
    \framesubtitle{Main Model}
    The paper explores how changes in industry-level import exposure affect labor outcomes at the commuting zone level, specifically analyzing unemployment rates, wages, income levels, and labor force participation.

    The primary estimation employs an Instrumental Variables Two-Stage Least Squares model\footnote{Clustered by area using FIP code.}:

    \begin{equation}
        \Delta L_{it}^{m} = \gamma_t + \beta_1 \Delta IPW_{it}^u + \Vec{X}_{it}'\Vec{\beta}_2 + \epsilon_{it}
        \label{eq:main_panel_regression}
    \end{equation}

    Where:

    \begin{itemize}
        \item $\Delta L_{it}^{m}$: 10-year change in the average share\footnote{Average calculated across groups based on gender, origin, age, education level, and ethnicity.} of manufacturing employment in the working-age population in commuting zone $i$ and year $t$.
        \item $\gamma_t$: Time fixed effect.
        \item $\Delta IPW_{it}^u$: Observed change in U.S. import exposure from China.
        \item $\Vec{X}_{it}$: Set of control variables for demographic and labor force characteristics.
    \end{itemize}
\end{frame}

%% ------------------------------------------------------------------------
\begin{frame}
    \frametitle{Methodology}
    \framesubtitle{Composition-Adjusted Model}
    The results of the main model are compared with those from a model in which labor outcomes are adjusted to control for changes in population composition. This alternative model is:

    \begin{equation}
        \Delta L_{it}^{CA, m} = \gamma_t + \beta_1 \Delta IPW_{it}^u + \Vec{X}_{it}'\Vec{\beta}_2 + \epsilon_{it}
        \label{eq:alternative_panel_regression}
    \end{equation}

    Where:

    \begin{itemize}
        \item $\Delta L_{it}^{CA, m}$: 10-year change in the composition-adjusted average share of manufacturing employment within the working-age population in commuting zone $i$ for year $t$.
    \end{itemize}
\end{frame}


%% ------------------------------------------------------------------------
\begin{frame}
    \frametitle{Methodology}
    \framesubtitle{Composition Adjustment}
    
    Original regression coefficients may be influenced by labor mobility between regions; therefore, composition adjustments are applied while aggregating the dependent variable by groups $g$. 
    
    In the case of the share of manufacturing employment within the working-age population $L_{igt}$, the composition adjusted average is defined as:

    \begin{equation}
        L_{it}^{CA} = \sum_g \bar{\theta}_{ig} L_{igt}
        \label{eq:composition_adjustment}
    \end{equation}

    Where the time-invariant weights $\bar{\theta}_{ig}$ are defined as:

    \begin{equation}
        \bar{\theta}_{ig} = \frac{\theta_{ig1990}+ \theta_{ig2000}+ \theta_{ig2008}}{3} 
        \label{eq:time_invariant_weight}
    \end{equation}
    
    and 
    \begin{equation}
        \theta_{igt} = \dfrac{hours_{igt}}{\sum_g hours_{igt}}.
        \label{eq:weight_per_group}
    \end{equation}

    %%Note that $\sum_g \bar{\theta}_{ig}=1$.
    Where $hours_{igt}$ denotes the number of weekly hours worked by group $g$ in commuting zone $i$ at time $t$.
\end{frame}
