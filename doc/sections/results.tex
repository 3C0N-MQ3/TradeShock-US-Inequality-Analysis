\begin{frame}
    \frametitle{Results}
    \framesubtitle{Impact of Chinese Imports on Manufacturing Employment Proportion (Table 3)}
    
    \begin{table}[ht]
        \centering
        \caption{Imports from China and Change of Manufacturing Employment in CZs, 1990--2007: 2SLS Estimates\footnote{Equivalent to \emph{Table 3} in the original paper, with additional columns for composition-adjusted models.}}
        \label{tab:table_3}
        \resizebox{1\textwidth}{!}{
            \begin{tabular}{lcccccccccccc}
                \toprule
                & \multicolumn{12}{c}{I. 1990--2007 stacked first differences} \\
                \cmidrule(lr){2-13}
                & \multicolumn{2}{c}{(1)} & \multicolumn{2}{c}{(2)} & \multicolumn{2}{c}{(3)} & \multicolumn{2}{c}{(4)} & \multicolumn{2}{c}{(5)} & \multicolumn{2}{c}{(6)} \\
                \cmidrule(lr){2-3} \cmidrule(lr){4-5} \cmidrule(lr){6-7} \cmidrule(lr){8-9} \cmidrule(lr){10-11} \cmidrule(lr){12-13}
                & Original    & CA         & Original    & CA         & Original    & CA         & Original    & CA         & Original    & CA         & Original    & CA         \\
                \midrule
$(\Delta \text{ imports from China to US})/\text{worker}$   
                & -0.746***   & -0.787***  & -0.610***   & -0.658***  & -0.538***   & -0.590***  & -0.508***   & -0.550***  & -0.562***   & -0.605***  & -0.596***   & -0.640***  \\
                & (0.068)     & (0.085)    & (0.094)     & (0.118)    & (0.091)     & (0.091)    & (0.081)     & (0.100)    & (0.096)     & (0.117)    & (0.099)     & (0.120)    \\
Percentage of employment in manufacturing$_{-1}$    
                &             &            & -0.035      & -0.034     & -0.052***   & -0.050***  & -0.061***   & -0.071***  & -0.056***   & -0.057***  & -0.040***   & -0.048***  \\
                &             &            & (0.022)     & (0.021)    & (0.020)     & (0.019)    & (0.017)     & (0.016)    & (0.016)     & (0.015)    & (0.013)     & (0.013)    \\
Percentage of college-educated population$_{-1}$    
                &             &            &             &            &            &             & -0.008      & -0.033*    &             &            & 0.013       & -0.008     \\
                &             &            &             &            &            &             & (0.016)     & (0.018)    &             &            & (0.012)     & (0.013)    \\
Percentage of foreign-born population$_{-1}$        
                &             &            &             &            &            &             & -0.007      & -0.006     &             &            & 0.030***    & 0.036***   \\
                &             &            &             &            &            &             & (0.008)     & (0.008)    &             &            & (0.011)     & (0.013)    \\
Percentage of employment among women$_{-1}$         
                &             &            &             &            &            &             & -0.054**    & -0.027     &             &            & -0.006      & 0.028      \\
                &             &            &             &            &            &             & (0.025)     & (0.029)    &             &            & (0.024)     & (0.030)    \\
Percentage of employment in routine occupations$_{-1}$   
                &             &            &             &            &             &            &             &            & -0.230***   & 0.113      & -0.245***   & -0.214     \\
                &             &            &             &            &             &            &             &            & (0.063)     & (0.309)    & (0.064)     & (0.247)    \\
Average offshorability index of occupations$_{-1}$  
                &             &            &             &            &             &            &             &            & 0.244       & -0.225***  & -0.059      & -0.247***  \\
                &             &            &             &            &             &            &             &            & (0.252)     & (0.072)    & (0.237)     & (0.072)    \\
Census division dummies                               
                & No          & No         & No          & No         & Yes         & Yes        & Yes         & Yes        & Yes         & Yes        & Yes         & Yes        \\
                \multicolumn{13}{l}{II. 2SLS first stage estimates} \\
                \midrule
$(\Delta \text{ imports from China to OTH})/\text{worker}$  
                & 0.792***    & 0.792***   & 0.664***    & 0.664***   & 0.652***    & 0.652***   & 0.635***    & 0.635***   & 0.638***    & 0.638***   & 0.631***    & 0.631***   \\
                & (0.079)     & (0.079)    & (0.086)     & (0.088)    & (0.090)     & (0.092)    & (0.090)     & (0.092)    & (0.087)     & (0.089)    & (0.087)     & (0.090)    \\
$R^2$          
                & 0.54        & 0.54       & 0.57        & 0.57       & 0.58        & 0.58       & 0.58        & 0.58       & 0.58        & 0.58       & 0.58        & 0.58       \\
                \bottomrule
            \end{tabular}
        }
        \vspace{0.2cm}
        
        \begin{minipage}{\linewidth}
            \tiny
            \textit{Notes:} $N = 1,444$ (722 commuting zones $\times$ 2 time periods). All regressions include a constant and a dummy for the 2000--2007 period. First stage estimates in panel II also include the control variables that are indicated in the corresponding columns of panel I. Routine occupations are defined such that they account for 1/3 of US employment in 1980. The offshorability index variable is standardized to mean of 0 and standard deviation of 10 in 1980. Robust standard errors in parentheses are clustered on state. Models are weighted by start of period CZ share of national population. \\
            ***Significant at the 1 percent level. \\
            **Significant at the 5 percent level. \\
            *Significant at the 10 percent level.
        \end{minipage}
    \end{table}
\end{frame}

%% ------------------------------------------------------------------------
%\begin{frame}
    %\frametitle{Results}
    %\framesubtitle{Impact of Chinese Imports on Manufacturing Employment Proportion (Summary)}
    %\begin{table}[h!]
        %\centering
        %\begin{adjustbox}{max width=\textwidth}
            %\begin{tabular}{c c c >{\raggedright\arraybackslash}p{8cm}}
                %\toprule
                %\multicolumn{4}{c}{\textbf{Coefficient ($\Delta$ Imports from China to US per Worker)}} \\
                %\midrule
                %\textbf{Model} & \textbf{Original} & \textbf{Adjusted} & \textbf{Description} \\
                %\midrule
                %1 & -0.746 & -0.787 & Demographic and labor force measures added to the first difference model for the period 1990–2007. \\
                %2 & -0.610 & -0.658 & Control for the share of manufacturing in a CZ's start-of-period employment. \\
                %3 & -0.538 & -0.590 & Geographic dummies for the nine Census divisions. \\
                %4 & -0.508 & -0.550 & Additional controls for education, foreign-born population, and working-age women employment. \\
                %5 & -0.562 & -0.605 & Control by \% manufacturing employment and variables capturing occupational susceptibility to technology and offshoring. \\
                %6 & -0.506 & -0.640 & Fully augmented model. \\
                %\bottomrule
            %\end{tabular}
        %\end{adjustbox}
        %\caption{Comparison of Regression Coefficients and Model Descriptions}
        %\label{tab:comparison_coeff}
    %\end{table}

    %The above table displays a summary of the outcomes from the original paper and those obtained when composition-adjusting while progressively adding controls.
%\end{frame}
%% ------------------------------------------------------------------------
\begin{frame}
    \frametitle{Results}
    \framesubtitle{Impact of Chinese Imports on Manufacturing Employment Proportion (Interpretation)}
    \begin{itemize}
        \item Composition-Adjusted estimates are consistently, albeit marginally, more negative than the original.
        \item In the model with all variables (6), the composition-adjusted $\beta = -0.640$ means that an exogenous increase of \$1,000.00 in per-worker exposure to Chinese imports leads to a predicted decrease of 0.64 percentage points in manufacturing employment per working-age population.
        \item The greater magnitude of the composition-adjusted estimates tells us that, had the composition of each CZ's population stayed fixed across time (aggregate to 128 sub-groups), exposure to exogenous Chinese import shocks would have resulted in a higher decrease in the share of working-age population employed in manufacturing.
    \end{itemize}
\end{frame}
%% ------------------------------------------------------------------------
\begin{frame}
    \frametitle{Import Shocks Effects on Employment Status}
        \begin{table}[ht]
        \centering
        \caption{Imports from China and Employment Status of Working-Age Population within CZs, 1990--2007: 2SLS Estimates\footnote{Equivalent to \emph{Table 5} in the original paper, with additional columns for composition-adjusted models. The full table is available in the appendix as Table \ref{tab:table_5}.}}
        \label{tab:table_5_panel_a}
        \resizebox{1\textwidth}{!}{
            \begin{tabular}{lcccccccccc}
                \toprule
                \multicolumn{9}{c}{Dependent variables: Ten-year equivalent changes in log population counts and population shares by employment status}&&\\
                \cmidrule(lr){1-11}
                & \multicolumn{2}{c}{Mfg emp} & \multicolumn{2}{c}{Non-mfg emp} & \multicolumn{2}{c}{Unemp} & \multicolumn{2}{c}{NILF} \\
                & \multicolumn{2}{c}{(1)}     & \multicolumn{2}{c}{(2)}         & \multicolumn{2}{c}{(3)}   & \multicolumn{2}{c}{(4)}  \\
                \cmidrule(lr){2-3} \cmidrule(lr){4-5} \cmidrule(lr){6-7} \cmidrule(lr){8-9}
                & Original & CA  & Original & CA  & Original & CA  & Original & CA \\
                \midrule
                \multicolumn{9}{l}{\textit{Panel A. 100 $\times$ log change in population counts}} \\
$(\Delta \text{ imports from China to US})/\text{worker}$ 
                & -4.231*** & -4.831*** & -0.274     & -0.013     & 4.921*** & 5.884*** & 2.058*   & 3.296*   \\
                & (1.047)   & (1.215)   & (0.651)    & (0.667)    & (1.128)  & (1.138)  & (1.080)  & (1.103)  \\
                \bottomrule
            \end{tabular}
        }
        \vspace{0.2cm}
        
        \begin{minipage}{\linewidth}
            \tiny
            \textit{Notes:} $N = 1,444$ (722 CZs $\times$ two time periods). All statistics are based on working age individuals (age 16 to 64). The effect of import exposure on the overall employment/population ratio can be computed as the sum of the coefficients for manufacturing and nonmanufacturing employment. All regressions include the full vector of control variables from column 6 of Table \ref{tab:table_3}. Robust standard errors in parentheses are clustered on state. Models are weighted by start of period CZ share of national population. \\
            ***Significant at the 1 percent level. \\
            **Significant at the 5 percent level. \\
            *Significant at the 10 percent level.
        \end{minipage}
    \end{table}
    The coefficients in this table represent the expected log change caused by Chinese import shocks in the number of working-age individuals in four categories of employment within CZs: 
        
    \begin{itemize}
        \item Manufacturing employment
        \item Non-manufacturing employment
        \item Unemployment
        \item Labor force non-participation
    \end{itemize}
\end{frame}

%% ------------------------------------------------------------------------


%% ------------------------------------------------------------------------
\begin{frame}
    \framesubtitle{Interpretation}
    
    \frametitle{Import Shocks Effects on Employment Status}
    The coefficients should be interpreted as the changes in the number of individuals per employment status in log points for every \$1,000 increase in import exposure per worker.
    E.g., in the original model, a \$1,000 per worker increase in import exposure reduces the number of workers in manufacturing employment by 4.2\% (vs. the 4.8\% of the composition-adjusted model).

    \begin{itemize}
        \item When composition-adjusting all estimates become more negative, except for the coefficient on non-manifacturing employment.
        \item Notably, the non-manufacturing employment coefficient is the only estimate to move closer to zero as a consequence of composition-adjusting. However, in both models this estimate remains statistically insignificant.
    \end{itemize}

    In conclusion, had the composition of each CZ stayed fixed through time we would have observed greater losses in manufacturing employment and greater hikes in unemployment and labor-force exits for same values of per-worker exposure to Chinese import shocks.
\end{frame}
%% ------------------------------------------------------------------------
%\begin{frame}
    %\frametitle{Wage Effects in Local Labor Markets by Gender}

        %\begin{table}[h!]
            %\centering
            %\begin{adjustbox}{max width=\textwidth}
                %\begin{tabular}{l c c c c c c}
                    %\toprule
                    %\multicolumn{7}{l}{Dependent variable: Ten-year equivalent change in average log weekly wage (in log pts)} \\
                    %\multicolumn{7}{l}{Regressor coefficient: ($\Delta$ imports from China to US) / worker} \\
                    %\midrule
                    %& \multicolumn{2}{c}{\textbf{All Workers}} & \multicolumn{2}{c}{\textbf{Males}} & \multicolumn{2}{c}{\textbf{Females}} \\
                    %\cmidrule(lr){2-3} \cmidrule(lr){4-5} \cmidrule(lr){6-7}
                    %& \textbf{Original} & \textbf{Adjusted} & \textbf{Original} & \textbf{Adjusted} & \textbf{Original} & \textbf{Adjusted} \\
                    %\midrule
                    %\textbf{All education levels} & -0.759*** & -1.222*** & -0.892*** & -1.204*** & -0.614*** & -1.258*** \\
                    %\textbf{College education} & -0.757*** & -0.903*** & -0.991*** & -1.129*** & -0.525*** & -0.598*** \\
                    %\textbf{No college education} & -0.814*** & -1.182*** & -0.703*** & -1.104*** & -1.116*** & -1.304*** \\
                    %\bottomrule
                %\end{tabular}
            %\end{adjustbox}
            %\caption{Ten-year equivalent change in average log weekly wage (in log pts) due to imports from China}
        %\end{table}

    %\end{frame}
%% ------------------------------------------------------------------------
\begin{frame}
    \frametitle{Impacts of Chinese Import Shocks on Wages by Gender and Edcation}
    \begin{table}[ht]
        \centering
        \caption{Imports from China and Wage Changes within CZs, 1990--2007: 2SLS Estimates\footnote{Equivalent to \emph{Table 6} in the original paper, with additional columns for composition-adjusted models.}}
        \label{tab:china_imports_wage_changes}
        \resizebox{0.9\textwidth}{!}{
            \begin{tabular}{lcccccccc}
                \toprule
                \multicolumn{7}{c}{Dependent variable: Ten-year equivalent change in average log weekly wage (in log pts)}&&\\
                \cmidrule(lr){2-9}
                & \multicolumn{2}{c}{All workers} & \multicolumn{2}{c}{Males} & \multicolumn{2}{c}{Females} \\
                & \multicolumn{2}{c}{(1)}         & \multicolumn{2}{c}{(2)}   & \multicolumn{2}{c}{(3)}     \\
                \cmidrule(lr){2-3} \cmidrule(lr){4-5} \cmidrule(lr){6-7}
                & Original & CA  & Original & CA  & Original & CA \\
                \midrule
                \multicolumn{9}{l}{\textit{Panel A. All education levels}} \\
$(\Delta \text{ imports from China to US})/\text{worker}$ 
                & -0.759*** & -1.222*** & -0.892*** & -1.203*** & -0.614*** & -1.258*** \\
                & (0.253)   & (0.290)   & (0.294)   & (0.329)   & (0.237)   & (0.285)   \\
$R^2$          
                & 0.56      & 0.56      & 0.44      & 0.43      & 0.69      & 0.68      \\
                \midrule
                \multicolumn{9}{l}{\textit{Panel B. College education}} \\
$(\Delta \text{ imports from China to US})/\text{worker}$ 
                & -0.757**  & -0.903*** & -0.991*** & -1.129*** & -0.525*   & -0.598*   \\
                & (0.308)   & (0.334)   & (0.374)   & (0.372)   & (0.279)   & (0.349)   \\
$R^2$          
                & 0.52      & 0.31      & 0.39      & 0.18      & 0.63      & 0.40      \\
                \midrule
                \multicolumn{9}{l}{\textit{Panel C. No college education}} \\
$(\Delta \text{ imports from China to US})/\text{worker}$ 
                & -0.814*** & -1.182*** & -0.703*** & -1.104*** & -1.116*** & -1.304*** \\
                & (0.236)   & (0.266)   & (0.250)   & (0.312)   & (0.278)   & (0.271)   \\
$R^2$          
                & 0.52      & 0.60      & 0.45      & 0.48      & 0.59      & 0.70      \\
                \bottomrule
            \end{tabular}
        }
        \vspace{0.2cm}
        
        \begin{minipage}{\linewidth}
            \tiny
            \textit{Notes:} $N = 1,444$ (722 CZs $\times$ two time periods). All regressions include the full vector of control variables from column 6 of Table \ref{tab:table_3}. Robust standard errors in parentheses are clustered on state. Models are weighted by start of period CZ share of national population. \\
            ***Significant at the 1 percent level. \\
            **Significant at the 5 percent level. \\
            *Significant at the 10 percent level.
        \end{minipage}
        \label{tab:table_6}
    \end{table}
\end{frame}

%% ------------------------------------------------------------------------
\begin{frame}
    \frametitle{Wage Effects in Local Labor Markets by Gender and Education}
    \framesubtitle{Interpretation}

    Composition-adjusting increases the magnitude of all coefficients, meaning that, had the population composition of CZs stayed fixed, we would have observed a greater decrease in wages regardless of gender and education level.
        
    E.g., a \$1,000 increase in per-worker import exposure is expected to result in a 0.991 log-point decrease in average weekly earnings for college educated males in the orginial model, compared to a 1.222 decrease for the same population sub-group in the composition-adjusted model.
    
    The rank order of coefficients is also mainted in the composition-adjusted model: 
    \begin{itemize}
        \item College-educated males are more negatively affected than those with no college education.
        
        \item Females with no college education are more negatively affected than those who attended college.
    
        \item Considering all workers regardless of gender, those with no college education are more negatively affected than those who attended college.
        \end{itemize}
        
\end{frame}